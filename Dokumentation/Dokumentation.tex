\documentclass[german, 11pt]{scrartcl}
\usepackage[a4paper]{geometry}
\geometry{verbose,tmargin=3.5cm,bmargin=2.0cm,lmargin=2.5cm,rmargin=2.5cm}

\usepackage[utf8]{inputenc}
\usepackage[ngerman]{babel}
\usepackage{amsmath,amssymb,amsfonts}
\usepackage{graphicx}
\usepackage{hyperref}
\usepackage{bm}                          % bold math symbols
\usepackage{xcolor}

\usepackage{float}                       % mit der option \begin{figure}[H] eine Grafik an eine feste Stelle bringen
\setlength\parindent{0pt}                % Im kompletten Text keine Einrueckung nach Absatz


%opening
\title{Dokumentation Aufblasprüfstand}
\author{}

\begin{document}

\maketitle

\section{Vorbereitung des PCs/Laptops}
Dieses Repository enthält alle notwendigen Dateien zum Aufblasprüfstand.
Sowohl die Software auf dem Mikrocontroller, als auch das Versuchsprogramm und die Auswertung basieren auf (Micro-) Python.
Deshalb muss auf dem PC zunächst Python installiert werden, worauf an dieser Stelle nicht weiter eingegangen wird.
Alle Pakete, die zusätzlich benötigt werden, sind in der Datei 'requirements.txt' aufgelistet
und können mit \textit{pip} installiert werden.
Es empfiehlt sich hierzu in dem vorliegenden Ordner zum Aufblasprüfstand eine virtuelle Python-Umgebung zu erstellen,
in welcher lediglich die benötigten Zusatzpakete installiert sind. Dies hat den Vorteil, dass man mehrere virtuelle Python-Umgebungen
mit ihren eigenen Zusatzpaketen und Versionsnummern auf einem PC haben kann.

\subsection{Virtuelle Python-Umgebung}
\begin{itemize}
    \item Ein Powershell-Fenster im Ordner 'Aufblasprüfstand' (Hauptordner des Repos) öffnen (Shift + rechte Maustaste)
    \item Eine virtuelle Python-Umgebung anlegen (nur einmal nötig):\\
    \texttt{python -m venv py\_venv} \hfill (hierdurch entsteht der Ordner 'py\_venv')
    \item Die virtuelle Python-Umgebung in der Powershell aktivieren (bei jedem Neustart der Powershell nötig):\\
    \texttt{py\_venv\textbackslash Scripts\textbackslash Activate.ps}
    \item Die notwendigen Zusatzpakete installieren (nur einmal nötig):\\
    \texttt{pip install -r requirements.txt}
\end{itemize}

\section{Hardware}

\subsection{Komponenten}
\begin{itemize}
    \item Raspberry Pi Pico
    \item Webcam Sandberg 133-97 Flex 1080P HD
    \item Drucksensor, 5V DC, 10psi
    \item 3/2-Wege Magnetventil, 12V DC
    \item Transistor XY
    \item Widerstände XY
    \item Freilaufdiode XY
\end{itemize}

Der Aufblasprüfstand produziert Daten, welche digital vorliegen und an den Laptop übertragen werden müssen.
Dies sind das Videosignal der Webcam sowie das Drucksignal des Drucksensors.\\

Während das Videosignal über die USB-Verbindung der Webcam zum PC übertragen wird,
ist für die Erfassung und Übertragung der Druckwerte ein Mikrocontroller zuständig (Raspberry Pi Pico).\\

Desweiteren wartet der Mikrocontroller auf Befehle des Laptops bzw. des Versuchsprogramms, um das Magnetventil zu schalten
und somit Luft zu- oder abzuführen.
Der Mikrocontroller muss somit sowohl Daten erfassen und ausgeben, und andererseits Daten einlesen.
Eine solche Aufgabe würde typischerweise ein Datenaufzeichnungsgerät (Data Aquisition/DAQ, z.B. von National Instruments) übernehmen,
auf welches aufgrund des Preises hier jedoch versucht wurde zu verzichten.\\

Um mit dem Mikrocontroller gleichzeitig Daten aus- und einlesen zu können, laufen auf ihm zwei Threads parallel.
Hierzu dient Micropython als Firmware auf dem Mikrocontroller, welches Multi-Threading erlaubt.

\subsection{Mikrocontroller flashen}
Diese Schritte sind nur notwendig, sofern ein neuer Pico vorliegt, oder eine neuere Version der Micropython-Firmware
auf dem Pico installiert werden soll:

\begin{itemize}
    \item Neuste Micropython-Firmware (Datei mit Endung '.uf2') herunterladen \footnote[1]{\url{https://micropython.org/download/rp2-pico/}}
    \item Die USB-Verbindung zwischen Pico und Laptop trennen und neu anschließen, dabei den weißen Knopf 'Bootsel' bis zum Einstecken des USB-Kabels gedrückt halten. Dadurch erscheint der Pico als neues USB-Gerät.
    \item Die heruntergeladene Firmware-Datei auf den Pico kopieren. Dies wird durch den Pico erkannt, wodurch er automatisch neu startet.
\end{itemize}

\subsection{Skript auf den Mikrocontroller laden}
Diese Schritte sind nur notwendig, sofern ein neuer Pico vorliegt, oder eine neuere Version des Hauptskriptes auf dem Pico installiert werden soll:
\begin{itemize}
    \item Pico per USB mit dem PC verbinden (ohne Drücken von 'Bootsel')
     \item \textit{optional:} COM-Port des Picos im Geräte-Manager heraussuchen
    \item Powershell-Fenster im Ordner 'Aufblasprüfstand' öffnen
    \item virtuelle Python-Umgebung aktivieren
    \item Remote Shell auf dem Pico ausführen:\\
    \texttt{rshell -p COM6} \hfill (den COM-Port des Picos benutzen, s.o.)
    \item Das Hauptskript auf den Pico kopieren:\\
    \texttt{cp pi\_pico/serial\_read/main.py /pyboard}
\end{itemize}

\subsection{Schaltung Mikrocontroller}
% Drucksensor
Der Drucksensor ist an die 3.3V Spannungsversorgung des Picos angeschlossen und gibt ein Analogsignal aus,
welches am Pico an Pin\footnote[2]{Pinbelegung Raspberry Pi Pico: \url{https://pico.pinout.xyz/}} 'A0' bzw. 'GP26' angeschlossen ist,
welcher ein ADC-Eingang (Analog-Digital-Converter) ist, an dem max. 3.3V anliegen dürfen. Die ADC-Eingänge des Picos arbeiten mit 16 bit (0-65535).
Die Kennlinie des Drucksensors ist so konzipiert, dass bei 0psi 10\% und bei 10psi 90\% seiner Versorgungsspannung anliegen:

\begin{equation*}
    \begin{aligned}
        v_{\text{in}} &= 3.3V, \quad V = \frac{sensorVal}{2^{16}}\,v_{\text{in}} \\
        v_0 &= \frac{1}{10}\,v_{\text{in}}, \quad v_{10} = \frac{9}{10}\,v_{\text{in}} \\
        m &= \frac{10\,\text{psi}}{v_{10}-v_0}, \quad b = -v_0 m \\
        p_{\text{psi}} &= m \cdot V + b, \quad p_{\text{mbar}} = 68.9476\,p_{\text{psi}}
    \end{aligned}
\end{equation*}

Der Sensorwert wird im Hauptskript auf dem Pico mit der Funktion \texttt{read\_u16()} ausgelesen.\\

% Magnetventil
Das Magnetventil wird mit einem separaten 12V-Netzteil bespeist. Um es jedoch mit den 3.3V-Pins des Picos beschalten zu können,
kann eine Transistorschaltung verwendet werden. Desweiteren ist eine Freilaufdiode notwendig, da beim Ausschalten des Magnetventils
eine induktive Last andererseits zu Spannungsspitzen in den dahinterliegenden Bauteilen führen würde.
Hierfür ist eine Freilaufdiode vom Typ XY ausreichend.\\

Die Transistorschaltung ist in Abb. XY gezeigt. Hierin ist ein Bipolartransistor (NPN) vom Typ XY verbaut.
Er besitzt eine Verstärkung von $\beta = XY$.
Zwischen Basis und Emitter fällt bei dieser Art von Transistor aufgrund der enthaltenen Diode(n) eine Spannung von 0.7V (?!) ab.
Steigt die Basis-Emitter-Spannung über einen bestimmten Wert, schaltet der Transistor zwischen Kollektor und Emitter durch.

\section{Software}
Das Versuchsprogramm 'Aufblaspruefstand\_main.py' stellt das Kamerabild inklusive des detektierten Luftballons
und dessen Durchmesser dar. Desweiteren trägt es die Druck- und Durchmesserwerte während der Messung über der Zeit auf.
Vor Beginn der Messung ermöglicht es, die Bildeinstellungen zur Objekterkennung zu justieren.
Am Ende der Messung korreliert es die Druck- und Durchmesserwerte zeitlich und speichert die Rohdaten sowie die korrelierten Daten separat ab.\\

Die Messung des Durchmessers geschieht durch Objekterkennung bzw. Auswertung des Kamerabildes. Hierzu wird \textit{opencv} verwendet.
Die Auswertung des Kamerabildes und die des Drucksignals arbeiten parallel in separaten Threads,
weshalb die Druck- und Durchmesserdaten zeitlich nicht automatisch korreliert sind.

Vielmehr wird die Taktung des Drucksignals im Hauptskript des Mikrocontrollers festgelegt.
Sobald der Mikrocontroller ein Drucksignal ausgibt, wird ein Event im Thread der Durchmessermessung getriggert
und die Uhrzeit des Events gespeichert.

Die Taktung des Drucksignals bzw. eine schnellere Taktung muss im Versuchsprogramm für die Durchmessermessung übernommen werden,
was in der Variable 'self.dt\_serial' geschieht. Das Versuchsprogramm wertet im vorgegebenen Takt das Kamerabild
hinsichtlich des Ballondurchmessers aus und speichert ebenfalls die jeweiligen Uhrzeiten.

Die Uhrzeiten der Druck- und der Durchmesserwerte werden typischerweise nicht identisch sein.
Mit Hilfe des Python-Pakets 'pandas' können beide Messreihen jedoch zeitlich korreliert werden.
Hierbei bleiben alle Druckwerte erhalten und alle Durchmesserwerte mit den geringsten zeitlichen Abständen
zu den Druckwerten, werden übernommen. Die Anzahl der Druck- und Durchmesserwerte ist im Anschluss identisch.
Bei dieser Art der zeitlichen Korrelation ist es also wichtig, die Durchmesserwerte öfter abzutasten, als die Druckwerte!

\section{Versuche}
\subsection{Vorbereitung}
\begin{itemize}
    \item Laptop einschalten, Webcam und Pico per USB mit dem Laptop verbinden, 12V-Netzteil für Magnetventil einstecken
    \item Druckluftschlauch ankuppeln, Eingangsdruck sicherstellen
    \item (Der Druckminderer ist fast geschlossen und somit so eingestellt, dass die Füllzeit des Luftballons nicht zu schnell ist.)
    \item manuelles Druckablassventil schließen
    \item Webcamhalterung auf den Tisch absenken und Gelenk arretieren/festschrauben
    \item Luftballon einspannen
    \item Versuchsprogramm starten:
    \begin{itemize}
        \item Powershell-Fenster im Ordner 'Aufblasprüfstand' (Hauptordner des Repos) öffnen (Shift + rechte Maustaste)
        \item virtuelle Python-Umgebung aktivieren: \texttt{py\_venv\textbackslash Scripts\textbackslash Activate.ps}
        \item Versuchsprogramm starten: \texttt{python Aufblaspruefstand\_main.py}
    \end{itemize}
    \item Ausrichtung des Aruco-Markers kontrollieren. Dieser muss in der Symmetrieebene des Luftballons liegen.
    \item Ausrichtung der Webcam kontrollieren. Hierzu das Bild im Versuchsprogramm beobachten.
    Der obere Bildrand sollte knapp über den breitesten Punkt/Durchmesser des Luftballons ragen.
    Der zylindrische Teil darf nicht sichtbar sein bzw. der Luftballon darf nicht als Kugel im Videobild zu sehen sein.
    Ansonsten erkennt die Bildauswertung ggf. die Höhe des Luftballons als dessen Breite/Durchmesser.\\
    Der untere Bildrand sollte knapp unterhalb des Aruco-Markers abschließen.
    In einem Abstand der Webcam von ca. XY mm zum Aruco-Marker sollte dies der Fall sein.
\end{itemize}

\subsection{Durchführung}
\begin{itemize}
    \item (Luftballon ggf. minimal mit dem Mund mit Luft füllen, sodass seine Kugelform sichtbar ist $\leadsto$ Durchmesserauswertung)
    \item Im Versuchsprogramm den Regler 'minArea' so weit runter regulieren, bis der Luftballon detektiert wird.
    \item Im Feld 'Zyklen' die anzufahrenden Durchmesser (komma-getrennt) eingeben, z.B. '120, 120, 150, 150, 200, 200'
    \item 'Messung starten' drücken
    \item Nachdem der letzte Zyklus durchlaufen wurde, manuell 'Messung beenden' drücken.\\
    Die Druck- und Durchmesserdaten werden daraufhin zeitlich korreliert und im Ordner 'Messungen' abgelegt.
\end{itemize}

\subsection{Auswertung}
Um die Daten für einen Plot vorzubereiten, können sie mit Hilfe eines GUI-Skriptes geglättet werden.
Dabei kommt ein Tiefpass- bzw. Butterworth-Filter (typischerweise dritter Ordnung) zum Einsatz,
welcher die Daten an der Grenzfrequenz auf das $\frac{1}{\sqrt{2}}$-fache des ursprünglichen Signals abschwächt.
\begin{itemize}
    \item Auswerteskript (GUI) öffnen:
    \begin{itemize}
        \item Powershell-Fenster im Ordner 'Aufblasprüfstand' (Hauptordner des Repos) öffnen (Shift + rechte Maustaste)
        \item virtuelle Python-Umgebung aktivieren: \texttt{py\_venv\textbackslash Scripts\textbackslash Activate.ps}
        \item Auswerteskript (GUI) öffnen: \texttt{python Auswertung\_tkinter.py}
    \end{itemize}
    \item Messung einladen (Datei 'Auswertung.txt')
    \item Schieberegler für die Glättung der Messwerte anpassen.
    \item Daten speichern/exportieren (selber Ordner, wie eingeladene Daten).
\end{itemize}

\end{document}
